\documentclass[manuscript]{style}
\ProvidesClass{style.cls}

\usepackage[backend=bibtex, style=alphabetic]{biblatex}
\addbibresource{references.bib}

\AtBeginDocument{%
  \providecommand\BibTeX{{%
    \normalfont B\kern-0.5em{\scshape i\kern-0.25em b}\kern-0.8em\TeX}}}

\setcopyright{rightsretained}
\copyrightyear{2020}
\acmYear{2020}


\acmConference[]{CSCI 6363}{Fall 2020}
%\citestyle{acmauthoryear}




%------------------------------------------------------------



\begin{document}


\title{Project Title}


\author{Ulvi Bajarani}
\affiliation{
    \institution{The University of Texas Rio Grande Valley:}
    \institution{Department of Information Technology}
}

\author{Askar Nurbekov}
\affiliation{
    \institution{The University of Texas Rio Grande Valley:}
    \institution{Department of Interdisciplinary Studies in Science and Technology}
}

\author{Daniel Ortiz}
\affiliation{
    \institution{The University of Texas Rio Grande Valley:}
    \institution{Department of Computer Science}
}

\author{Carlos Espinosa}
\affiliation{
    \institution{The University of Texas Rio Grande Valley:}
    \institution{Department of Information Technology}
}

\author{Mariana Martinez}
\affiliation{
    \institution{The University of Texas Rio Grande Valley:}
    \institution{Department of Computer Science}
}


\renewcommand{\shortauthors}{U. Bajarani, A. Nurbekov, D. Ortiz, C. Espinosa \& M. Martinez}


\begin{abstract}
Games such as chess where there needs to be two players, can become a problem when there isn’t a second person available to play as we have seen in these times of isolation due to COVID-19. Inspired by other robotic chess playing agents, we propose a portable, with less additional equipment robot using an ubiquitous approach to the board as well as offering an expansion for other board games and a coaching component.
\end{abstract}


\keywords{Artificial Intelligence, robotics, image processing, computer vision}


\maketitle


\section*{Introduction}

Test \cite{chen2019robust} Test2 \cite{rath2019autonomous}

\section*{Proposed Methods}



\section*{Anticipated Results}




\printbibliography


\end{document}
\endinput

