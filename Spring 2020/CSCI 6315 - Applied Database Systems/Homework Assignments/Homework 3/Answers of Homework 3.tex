\documentclass[paper=8.27in:11.69in, 14pt, DIV=calc]{scrartcl}
\usepackage{geometry}
\usepackage{graphicx}
\usepackage{pdfpages}
\usepackage{hyperref}
\usepackage{enumitem}
\usepackage{amsmath}


\begin{document}

\textbf{\begin{center}
\begin{Large}
CSCI 6315 Applied Database Systems\\
ASSIGNMENT 3: Functional Dependencies and Normalization\\
02/27/2020 00:30\\
Ulvi Bajarani\\
Student ID 20539914\\
E-mail: ulvi.bajarani01@utrgv.edu\\
\end{Large}
\end{center}}

\newpage

\begin{center}
\begin{Large}
\noindent \textbf{Initial Relationship R:}\\

\hfill

\noindent $R \ = \ (A, B, C, D, E)$\\

\textbf{\\Initial set F of Functional Dependencies holds on R:}\\

\hfill

$A \rightarrow BC$\\
$CD \rightarrow E$\\
$B \rightarrow D$\\
$E \rightarrow A$\\

\end{Large}
\end{center}
\newpage

\noindent \begin{center}
\textbf{Answers:}
\end{center}

\begin{enumerate}[label=\arabic*)]

\item \textit{Suppose that we decompose the schema $R$ into $R_{1} \ = \ (A, B, C)$ and $R_{2} \ = \ (A, D, E)$. Show that this decomposition is a lossless-join decomposition with respect to $F$.\\}

To understand if the decomposition is a lossless-join, we should understand if $A$ has a functional dependency with either $B$ or $C$. It helps do define if $A$ is a superkey or not.\\

Knowing $A \rightarrow BC$, we might also define $A \rightarrow B$ and $A \rightarrow C$ by the decomposition rule. Then, by transitivity we define that $A \rightarrow D$, and as a result, $AC \rightarrow E$, which might be reduced to $A \rightarrow E$.\\

Knowing $R_{1} \cap R_{2} = (A, B, C) \cap (A, D, E) = (A)$, $A \rightarrow A, B, C, D, E$ it is defined that $A \rightarrow R_{1}$. It proves that the decomposition is a lossless-join decomposition with respect to $F$.

\newpage
\item \textit{Suppose that we decompose the schema $R$ into $R_{1} \ = \ (A, B, C)$ and $R_{2} \ = \ (C, D, E)$. Show that this decomposition is not a lossless-join decomposition.\\}

We use the same approach and the functional dependencies as in the Problem 1:\\

To understand if the decomposition is a lossless-join, we should understand if $C$ has a functional dependency with either $D$ or $E$. After using calculations in Problem 1, it might be seen that $C$ is not enough to imply $E$ alone: only with a combination in $D$, $AD \rightarrow E$.\\

Knowing $R_{1} \cap R_{2} = (A, B, C) \cap (C, D, E) = (C)$, $C \rightarrow D$ it is defined that $C \not \rightarrow R_{1}$ and $C \not \rightarrow R_{2}$. It proves that the decomposition is \textbf{not} a lossless-join decomposition with respect to $F$.

\newpage
\item \textit{Compute $(BC)^{+}$.\\}

The steps:

\begin{enumerate}[label=\arabic*)]

\item $(BC)^{+}\Rightarrow \ (BC)$
\item $(BC)^{+}\Rightarrow \ (BCD)$ due to $B \rightarrow D$
\item $(BC)^{+}\Rightarrow \ (BCDE)$ due to $B \rightarrow D$ and $CD \rightarrow E$
\item $(BC)^{+}\Rightarrow \ (ABCDE)$ due to $E \rightarrow A$

\end{enumerate}

\textbf{The answer is: $(BC)^{+}\Rightarrow \ (ABCDE)$}

\newpage
\item \textit{Compute the canonical cover $F_{c}$ .\\}

$F_{c} = (A \rightarrow B, A \rightarrow C, CD \rightarrow E, B \rightarrow D, E \rightarrow A)$\ which are \\ $F_{c} \rightarrow F$ and $F \rightarrow F_{c}$

\newpage
\item \textit{Show that the decomposition of $R$ into $R_{1} \ = \ (A, B, C)$ and $R_{2} \ = \ (A, D, E)$ is not a dependency-preserving decomposition.\\}

Because of the fact that $B$ and $D$ are in different tables, the decomposition could not be a dependency-preserving.

\newpage
\textbf{For Problem 6 and Problem 7, we have to define candidate keys for the relationships. They are $A$, $E$, $CD$, $BC$.}

\item \textit{Give a lossless-join decomposition into BCNF of $R$.\\}

In the functional dependency $B \rightarrow D$, $B$ is not a superkey, so we have to decompose $R \ = \ (A, B, C, D, E)$ into $R_{0} \ = \ (B, D)$ and $R_{1} \ = \ R \ - \ D = \ (A, B, C, E) $. While $R_{1}$ is in BCNF form, $R_{2}$ is not due to $A \rightarrow BC$, where $A$ is not a superkey. So, we have to decompose $R_{1} \ = \ (A, B, C, E)$ into $R_{2} \ = \ (A, B, C)$ and $R_{3} \ = \ R_{1} \ - \ (B, C) = \ (A, E) $. Finally, all tables are in BCNF form.

\textbf{The answer is:}

$R_{0} \ = \ (A, B, C)$ that has $F^{+}_{0} = (A \rightarrow B , A \rightarrow C)$.\\
$R_{1} \ = \ (B, D)$ that has $F^{+}_{1} = (B \rightarrow D)$.\\
$R_{2} \ = \ (A, E)$ that has $F^{+}_{2} = (E \rightarrow A)$.\\

\newpage
\item \textit{Give a lossless-join, dependency preserving decomposition into 3NF of $R$.\\}

Knowing $F_{c} = (A \rightarrow B, A \rightarrow C, CD \rightarrow E, B \rightarrow D, E \rightarrow A)$\\
\hfill
We might create tables:

$R_{0} \ = \ (A, B)$ that has $F^{+}_{0} = (A \rightarrow B)$.\\
$R_{1} \ = \ (A, C)$ that has $F^{+}_{1} = (A \rightarrow C)$.\\
$R_{2} \ = \ (C, D, E)$ that has $F^{+}_{2} = (CD \rightarrow E)$.\\
$R_{3} \ = \ (B, D)$ that has $F^{+}_{3} = (B \rightarrow D)$.\\
$R_{4} \ = \ (E, A)$ that has $F^{+}_{4} = (E \rightarrow A)$.\\

Due to the fact that the key $BC$ doesn't occur in the relations, we have to create 

$R_{5} \ = \ (B, C, D)$ that has $F^{+}_{5} = (BC \rightarrow D)$, because $(A \rightarrow B)$, $(B \rightarrow D)$, and $(A \rightarrow C)$. Also, we might composite it with $R_{2}$

We might simplify $R_{0}$ and $R_{1}$ as $(A, B, C)$

\textbf{The answer is:}

$R_{0} \ = \ (A, B, C)$ that has $F^{+}_{0} = (A \rightarrow B, A \rightarrow C)$.\\
$R_{1} \ = \ (C, D, E)$ that has $F^{+}_{1} = (CD \rightarrow E)$.\\
$R_{2} \ = \ (B, C, D)$ that has $F^{+}_{2} = (BC \rightarrow D)$.\\
$R_{3} \ = \ (E, A)$ that has $F^{+}_{3} = (E \rightarrow A)$.\\

\end{enumerate}

\end{document}