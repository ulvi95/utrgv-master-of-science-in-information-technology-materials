\documentclass[paper=8.2in:11.6in]{scrartcl}
\usepackage{geometry}
\usepackage{enumitem}
\usepackage{fancyhdr}
\pagestyle{fancy}

\begin{document}

\chead{N2 questions from Ulvi Bajarani, Student ID 20539914}

\begin{enumerate}

\item Which of them might be the registers of 16-bit processors?

\begin{enumerate}[label=\arabic*.]

\item EAX.
\item AX.
\item AH.
\item AL.

\end{enumerate}

\begin{enumerate}[label=\alph*)]

\item 1, 2, 3, 4.
\item 2, 3, 4.
\item 2.
\item 3, 4.

\end{enumerate}

\item Which of them might be the registers of 32-bit processors?

\begin{enumerate}[label=\arabic*.]

\item ECX.
\item CX.
\item CH.
\item CL.

\end{enumerate}

\begin{enumerate}[label=\alph*)]

\item 1, 2, 3, 4
\item 2, 3, 4
\item 2
\item 3, 4

\end{enumerate}

\item The advantage of Accumulator type of ISA (Instruction Set Architecture):

\begin{enumerate}[label=\alph*)]

\item Simple model of expression execution.
\item Data can be stored for a long time.
\item The shortness of instructions.

\end{enumerate}

\item The advantage of Stack type of ISA (Instruction Set Architecture):

\begin{enumerate}[label=\Roman*.]

\item Simple model of expression execution.
\item Data can be stored for a long time.
\item The shortness of instructions.

\end{enumerate}

\begin{enumerate}[label=\alph*)]

\item I, III.
\item I, II, III.
\item III.

\end{enumerate}

\item The advantage of GPR (General Purpose Registers) of ISA (Instruction Set Architecture):

\begin{enumerate}[label=\Roman*.]

\item Simple model of expression execution.
\item Data can be stored for a long time.
\item The shortness of instructions.

\end{enumerate}

\begin{enumerate}[label=\alph*)]

\item II, III.
\item II.
\item I, III.

\end{enumerate}

\item Choose the right numbers to the place of three dots (\ldots):

A total number of Simplified CPU registers is \ldots :

\begin{enumerate}[label=\alph*)]

\item 32
\item 33
\item 34
\item 35

\end{enumerate}

\item The disadvantage of Accumulator type of ISA (Instruction Set Architecture):

\begin{enumerate}[label=\Roman*.]

\item All operators must go through the accumulator.
\item Data cannot be stored for a long time.
\item Memory traffic for this approach is higher than others.

\end{enumerate}

\begin{enumerate}[label=\alph*)]

\item I, III.
\item I, II, III.
\item I, II.

\end{enumerate}

\item The disadvantage of Stack type of ISA (Instruction Set Architecture):

\begin{enumerate}[label=\Roman*.]

\item Only data from the top of stack can be accessed.
\item Every operation is executed in the stack.
\item All operators must be named.

\end{enumerate}

\begin{enumerate}[label=\alph*)]

\item I, II.
\item I, II, III.
\item II, III.

\end{enumerate}

\item The disadvantage of GPR (General Purpose Registers) of ISA (Instruction Set Architecture):

\begin{enumerate}[label=\Roman*.]

\item Memory traffic for this approach is higher than others.
\item Data cannot be stored for a long time.
\item All operators must be named.

\end{enumerate}

\begin{enumerate}[label=\alph*)]

\item II, III.
\item III.
\item I, III.

\end{enumerate}

\item Choose the right problems during the engineering of ISA:

\begin{enumerate}[label=\arabic*.]

\item Where should operands be stored?
\item How many explicit operands should be created?.
\item What operations should be provided by the ISA?
\item The type of operands and their size.

\end{enumerate}

\begin{enumerate}[label=\alph*)]

\item 1, 2, 3, 4.
\item 1, 3, 4.
\item 1, 2, 3.
\item 2, 3, 4.

\end{enumerate}

\end{enumerate}

\newpage

\chead{N2 Answers from Ulvi Bajarani, Student ID 20539914}

Answers:

\begin{enumerate}

\item b
\item a
\item c
\item a
\item b
\item d
\item b
\item a
\item b
\item a

\end{enumerate}



\end{document}